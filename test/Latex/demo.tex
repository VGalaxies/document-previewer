\documentclass[UTF8]{ctexart}

\title{文档预览器}
\author{刘添翼}
\date{\today}

\usepackage{indentfirst}
\setlength{\parindent}{2em}

\usepackage{hyperref}
\hypersetup{
	colorlinks=true,
}

\usepackage{listings}
\lstset{
	columns=fixed,       
	numbers=left,     % 在左侧显示行号
	language=java,    % 设置语言
}

\usepackage{graphicx}
\usepackage{float}

\begin{document}
\maketitle
\tableofcontents
\section{最后期限}
Project2的最后期限是\textbf{北京时间2021年11月14日23时59分}。


在此之前,每位同学需要提交课程项目的源码、可执行文件和项目报告到...

\section{技术指标}

\subsection{对Microsoft Word文档的支持}
借助已经存在的开源的\emph{C++}库...

\paragraph{问题2.1.}
我对\underline{Microsoft Word}和XML都完全不熟悉,我要怎么办?

这里有一个超链接\href{https://cn.bing.com/}{必应},STFW。

\begin{itemize}
    \item Fedora
    \item Fedora Spin
    \item Fedora Silverblue
\end{itemize}

\begin{enumerate}
    \item Fedora CoreOS
    \item Fedora Silverblue
    \item Fedora Spin
\end{enumerate}

\begin{lstlisting}
public class Demo {
    public static int binaryOne(int n) {
        if (n <= 0)
            throw new RuntimeException();

        if (n == 1) {
            return 1;
        } else if (n % 2 == 0) {
            return binaryOne(n / 2);
        } else {
            return 1 + binaryOne(n / 2);
        }
    }


    public static void main(String[] args) {
        binaryOne(3);
    }
}
\end{lstlisting}

\begin{figure}[H]
	\centering
	\includegraphics{img/88833228_p0.jpg}
\end{figure}

\begin{equation}
J(x)=L i(x)-\sum_{\rho} L i\left(x^{\rho}\right)-\ln 2+\int_{0}^{\infty} \frac{d t}{t\left(t^{2}-1\right) \ln t}
\end{equation}

\end{document}